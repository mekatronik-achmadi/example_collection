\documentclass[table,dvipsnames]{beamer}
\mode<presentation>{
	\usetheme{Madrid}
	\setbeamercolor{title}{fg=Black,bg=Blue!15}
	\setbeamercolor{frametitle}{fg=Black,bg=Blue!15}
	\setbeamercolor{block title}{fg=Black,bg=Blue!15}
	\setbeamercolor{block}{fg=Black,bg=Blue!10}
}

%\usepackage{default}
\usepackage{graphicx}
\usepackage{booktabs}
\usepackage{xcolor}
\usepackage{multirow}
\usepackage{minted}
\usepackage{keystroke}
\usepackage{menukeys}
\usepackage[
type={CC},
modifier={by-sa},
version={4.0},
]{doclicense}

\definecolor{LightGray}{gray}{0.9}

\title[Vim Cheatsheet]{Vim Summary Cheatsheet}
\author{}
\date{}

\begin{document}
    \section{Start}
    \begin{frame}
        \titlepage
    \end{frame}

    \begin{frame}
        \frametitle{Document License}
		\begin{exampleblock}{}
			\doclicenseThis
		\end{exampleblock}

		\begin{exampleblock}{}
			You are free to:
			\begin{itemize}
				\item Share: copy and redistribute the material in any medium or format
				\item Adapt: remix, transform, and build upon the material
				for any purpose, even commercially.
			\end{itemize}
		\end{exampleblock}

		\begin{exampleblock}{}
			Under the following terms:
			\begin{itemize}
				\item Attribution: you must give appropriate credit and indicate changes.
				\item ShareAlike: remix, transform, or build upon the material must under the same license as the original.
			\end{itemize}
		\end{exampleblock}
	\end{frame}

    \section{Vimtutor Summary}
    \begin{frame}
        \frametitle{Vimtutor: Lesson-1}
        \begin{exampleblock}{}
            \textbf{h}($\leftarrow$) \textbf{j}($\downarrow$)
            \textbf{k}($\uparrow$) \textbf{l}($\rightarrow$)
            = \textbf{h} and \textbf{l} in most left and right, \textbf{j} like down arrow.
        \end{exampleblock}

        \begin{exampleblock}{}
            \textbf{vim FILENAME} \keys{\return} = open a file from shell
        \end{exampleblock}
    
    	\begin{exampleblock}{}
    		\Esc [\textbf{cmd}\textbar \textbf{:cmd}] \keys{\return} = exit to NORMAL mode then input a command
    	\end{exampleblock}

        \begin{exampleblock}{}
            \textbf{:q!} = ignore changes and quit\\
    		\textbf{:wq} = write changes and quit
    	\end{exampleblock}
    
    	\begin{exampleblock}{}
    		\textbf{x} = delete a character at the cursor
    	\end{exampleblock}
    
    	\begin{exampleblock}{}
    		\textbf{i} = INSERT mode at the cursor
    	\end{exampleblock}
    
    	\begin{exampleblock}{}
    		\textbf{A} = INSERT mode at end of line
    	\end{exampleblock}
    \end{frame}

	\begin{frame}
		\frametitle{Vimtutor: Lesson-2}
		\begin{exampleblock}{}
			\textbf{dw} = delete from cursor to next word
		\end{exampleblock}
	
		\begin{exampleblock}{}
			\textbf{d\$} = delete from cursor to end of line
		\end{exampleblock}
	
		\begin{exampleblock}{}
			\textbf{dd} = delete whole line
		\end{exampleblock}
	
		\begin{exampleblock}{}
			[number] motion = \textbf{2w} = move cursor two words
		\end{exampleblock}	
	
		\begin{exampleblock}{}
			operator [number] motion = \textbf{d2w} = delete two words
		\end{exampleblock}
	
		\begin{exampleblock}{}
			\textbf{0} = move cursor to start of line
		\end{exampleblock}	
	
		\begin{exampleblock}{}
			\textbf{u} = undo last action\\
			\textbf{U} = undo all actions on whole line\\
			\keys{\ctrl}+\textbf{r} = redo last undo action
		\end{exampleblock}
	\end{frame}

	\begin{frame}
		\frametitle{Vimtutor: Lesson-3}
		\begin{exampleblock}{}
			\textbf{p} = put deleted text after cursor
		\end{exampleblock}
	
		\begin{exampleblock}{}
			\textbf{r} [char] = replace the character on cursor
		\end{exampleblock}
	
		\begin{exampleblock}{}
			\textbf{c} [motion] = change characters until motion\\
			\textbf{cw} = change characters until next word\\
			\textbf{ce} = change characters until end of word\\
			\textbf{c\$} = change characters until end of line
		\end{exampleblock}
	
		\begin{exampleblock}{}
			\textbf{c}[number]motion = change characters until counted motion\\
			\textbf{c2w} = change characters until 2 of next word\\
			\textbf{c2e} = change characters until 2 of end of word
		\end{exampleblock}
	\end{frame}

	\begin{frame}
		\frametitle{Vimtutor: Lesson-4}
		\begin{exampleblock}{}
			\textbf{G} = move to end line of file\\
			\textbf{gg} = move to start line of file\\
			X\textbf{G} = move to line X number of file
		\end{exampleblock}
	
		\begin{exampleblock}{}
			\textbf{/}foo = search foo forward\\
			\textbf{?}foo = search foo backward\\
			\textbf{n} = go to next search occurence\\
			\textbf{N} = go to previous search occurence
		\end{exampleblock}
	
		\begin{exampleblock}{}
			\textbf{\%} = highlight matching bracket\\
		\end{exampleblock}
	
		\begin{exampleblock}{}
			\textbf{:s/foo/bar} = substitute first foo to bar in line\\
			\textbf{:s/foo/bar/g} = substitute all foo to bar in line\\
			\textbf{:\%s/foo/bar/g} = substitute all foo to bar in file
		\end{exampleblock}
	\end{frame}

	\begin{frame}
		\frametitle{Vimtutor: Lesson-5}
		\begin{exampleblock}{}
			\textbf{:!cmd} = run external command
		\end{exampleblock}
	
		\begin{exampleblock}{}
			\textbf{:w} [FILENAME] = write all text to FILENAME
		\end{exampleblock}
	
		\begin{exampleblock}{}
			\Esc v [motion] \textbf{:w} [FILENAME] \\
			= write visually selected text to FILENAME
		\end{exampleblock}
	
		\begin{exampleblock}{}
			\textbf{:r} FILENAME = read a FILENAME and put it's text at cursor\\
			\textbf{:r} !cmd = read a command output and put it's text at cursor
		\end{exampleblock}
	\end{frame}

	\begin{frame}
		\frametitle{Vimtutor: Lesson-6}
		\begin{exampleblock}{}
			\textbf{o} = open line below cursor and start INSERT mode\\
			\textbf{O} = open line above cursor and start INSERT mode
		\end{exampleblock}
	
		\begin{exampleblock}{}
			\textbf{a} = start INSERT mode at after cursor\\
			\textbf{A} = start INSERT mode at end of line
		\end{exampleblock}
	
		\begin{exampleblock}{}
			\Esc v [motion] = visually selected text\\
			\textbf{y} = yank (copy) the selected text\\
			\textbf{p} = put copied text at cursor
		\end{exampleblock}
	
		\begin{exampleblock}{}
			\textbf{R} = replace characters until exit to NORMAL mode (\Esc)
		\end{exampleblock}
	
		\begin{exampleblock}{}
			\Esc \textbf{:q} = Quit Vim
		\end{exampleblock}
	\end{frame}

	\section{Vim Commands}
	\begin{frame}
		\frametitle{Commands-1}
	
		\begin{exampleblock}{}
			\Esc \textbf{:q} = Quit Vim\\
			\Esc \textbf{q:} = Command History
		\end{exampleblock}
	
		\begin{exampleblock}{}
			\textbf{gt} = switch to next tab\\
			\textbf{gT} = switch to previous tab
		\end{exampleblock}
	
		\begin{exampleblock}{}
			\textbf{:NERDTree} = Open File List window\\
			\keys{\return} = Open File or Toggle Fold Directory\\
			\textbf{r} = refresh file list\\
			\textbf{t} = open file in new tab\\
			\textbf{I} = toggle show hidden files\\
			\textbf{q} = close file list window
		\end{exampleblock}

	\end{frame}

	\begin{frame}
		\frametitle{Commands-2}
		\begin{exampleblock}{}
			\textbf{:TagbarToggle} = open/close tag list window
		\end{exampleblock}
		
		\begin{exampleblock}{}
			\keys{\ctrl}+\textbf{w} \keys{\ctrl}+\textbf{w} = switch window\\
			\keys{\ctrl}+\textbf{w} \textbf{l} = switch window right\\
			\keys{\ctrl}+\textbf{w} \textbf{h} = switch window left
		\end{exampleblock}
	
		\begin{exampleblock}{}
			\keys{\ctrl}+\textbf{w} \textbf{$<$} = decrease window size\\
			\keys{\ctrl}+\textbf{w} \textbf{$>$} = increase window size\\
			\textbf{30} \keys{\ctrl}+\textbf{w} \textbf{$|$} = set window size at 30
		\end{exampleblock}
		
		\begin{exampleblock}{}
			\textbf{:cmd} \keys{\ctrl}+\textbf{d} = autocomplete a command\\
			\textbf{:cmd} \keys{\tab} = autocomplete a command
		\end{exampleblock}
	
	\end{frame}

	\begin{frame}
		\frametitle{Commands-3}

		\begin{exampleblock}{}
			\textbf{:\%retab} = replace all tab with 4 spaces in file
		\end{exampleblock}
	
		\begin{exampleblock}{}
			\keys{$\backslash$} = default leader key
		\end{exampleblock}
		
		\begin{exampleblock}{}
			\Esc v = select text to comment\\
			\keys{$\backslash$}\textbf{cc} = comment line\\
			\keys{$\backslash$}\textbf{cn} = nested comment block\\
			\keys{$\backslash$}\textbf{c}[\textbf{space}] = toggle comment block
		\end{exampleblock}
	
		\begin{exampleblock}{}
			\textbf{ds} [char] = delete surrounding character\\
			\textbf{dst} = delete surrounding tag\\
			\textbf{cs} [old] [new] = change surrounding old to new\\
			\textbf{ys} [motion] char = add surrounding char from cursor to motion
		\end{exampleblock}
	\end{frame}

\end{document}
